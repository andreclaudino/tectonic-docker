\usepackage[font=default]{./repUERJ/repUERJformat} % pacote para as 
                                                  % normas da UERJ
%\usepackage[alf]{abntex2cite} % pacote para citacoes
\usepackage[brazil]{babel}  % adequação para o português Brasil
\usepackage[utf8]{inputenc} % Determina a codificação utilizada
                            % (conversão automática dos acentos)
\usepackage{makeidx}        % Cria o índice
\usepackage{hyperref}       % Controla a formação do índice
\usepackage{indentfirst}    % Indenta o primeiro paragrafo de
                            % cada seção.
\usepackage{graphicx}       % Inclusão de gráficos
\usepackage{subfig}
\usepackage{multirow}
\usepackage{amssymb}  % pacotes matemáticos
\usepackage{tikz}
\usepackage{./repUERJ/kbordermatrix}

\usepackage[dots=yes,num=yes]{./repUERJ/repUERJpseudocode}

\usepackage[maxfloats=25]{morefloats}
\usepackage{array}
\usepackage{cite}

\usepackage[bookmarks]{hyperref}
\usepackage{float}

\usepackage{caption}